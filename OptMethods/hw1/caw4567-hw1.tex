\documentclass[11pt]{article}

\usepackage{thumbpdf, amssymb, amsmath, amsthm, microtype,
	    graphicx, verbatim, listings, color, fancybox}
\usepackage[pdftex]{hyperref}
%\usepackage[margin=1in]{geometry}
\usepackage{cawsty}
\usepackage{fullpage}
\usepackage{pseudocode}

%\setlength{\parindent}{0pt}

\linespread{1.2}

\begin{document}
\arltitle{4040-849 Optimization Methods}{Written Assignment 1}
\begin{prob}{1}
\end{prob}
\begin{sol} 

The information provided about the mixed nuts, their content percentage, and cost can be displayed in tabular form as follows: \\

\begin{center}
\begin{tabular}{|l|l|l|}
\hline
\textbf{Nut} & \textbf{Type A (\%)} & \textbf{Type B (\%)} \\
\hline
Almonds & 0.20 & 0.10 \\
Cashew nuts & 0.10 & 0.20 \\
Walnuts & 0.15 & 0.25 \\
Peanuts & 0.55 & 0.45 \\
\hline
\textbf{Cost} & \$2.50 & \$3.00 \\
\hline
\end{tabular}  \\
\end{center}

From this information we can determine the following properties of the problem: \\

\textbf{Design variables:}\\
\begin{eqnarray*}
x = \text{ the number of pounds of mixed nut type} A \\
y = \text{ the number of pounds of mixed nut type} B
\end{eqnarray*}

\textbf{Constraints:} \\
\begin{eqnarray*}
0.2x + 0.1y \geq 4 \\
0.1x + 0.2y \geq 5 \\
0.15x + 0.25y \geq 6
\end{eqnarray*}

\textbf{Cost function:}
\begin{eqnarray*}
f(x,y) = 2.5x + 3y
\end{eqnarray*}

TODO: convert into standard form now... (equality with constraints, etc..)
\end{sol}

\begin{prob}{2}
\end{prob}
\begin{sol} 

Based on the problem description, we can identify the following properties of the problem: \\

\textbf{Design variables:}\\ 
\begin{eqnarray*}
x_{1,1} = \text{ \#resources from New York to Seattle}\\
x_{1,2} = \text{ \#resources from New York to Houston} \\
x_{1,3} = \text{ \#resources from New York to Detroit} \\
x_{2,1} = \text{ \#resources from Los Angeles to Seattle} \\
x_{2,2} = \text{ \#resources from Los Angeles to Houston} \\
x_{2,3} = \text{ \#resources from Los Angeles to Detroit}
\end{eqnarray*}

\textbf{Constraints:} \\
\begin{eqnarray*}
x_{1,1} + x_{1,2} + x_{1,3} \leq 3 \\
x_{2,1} + x_{2,2} + x_{2,3} \leq 3 \\
x_{1,1} + x_{2,1} = 2 \\
x_{1,2} + x_{2,2} = 3 \\
x_{1,3} + x_{2,3} = 1 
\end{eqnarray*}

\textbf{Cost function:}
\begin{eqnarray*}
f(x_{1,1},x_{1,2},x_{1,3},x_{2,1},x_{2,2},x_{2,3}) = 4x_{1,1} + 3x_{1,2} + x_{1,3} + 2x_{2,1} + 7x_{2,2} + 5x_{2,3}
\end{eqnarray*}

TODO: convert into standard form now... (equality with constraints, etc..)

\end{sol}

\begin{prob}{3}
\end{prob}
\begin{sol} 
A customer will select a variable amount of CDs from the selections provided in order to maximize their return. Therefore, the design variables for this problem should be the number of CDs for each type available, namely, $c_{1}, c_{2}, c_{3},$ and $c_{4}$, for types $1, 2, 3,$ and $4$, respectively. Using these variables and a total of \$50,000 in investment money, the objective (cost) function that can be maximize to yield the total revenue can be defined as follows:

\begin{equation*}
f(c_{1}, c_{2}, c_{3}, c_{4}) = 2500c_{1} + 3500c_{2} + 5000c_{3} + 7500c_{4}
\end{equation*}

Based on the duration for each type of CD, we can identify the following constraints 

\end{sol}

\end{document}
