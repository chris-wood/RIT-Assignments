\documentclass[paper=a4, fontsize=11pt]{scrartcl} % A4 paper and 11pt font size
\usepackage[left=2cm,top=1cm,right=2cm,nohead,nofoot]{geometry}
\usepackage[T1]{fontenc} % Use 8-bit encoding that has 256 glyphs
\usepackage{fourier} % Use the Adobe Utopia font for the document - comment this line to return to the LaTeX default
\usepackage[english]{babel} % English language/hyphenation
\usepackage{amsmath,amsfonts,amsthm} % Math packages
\usepackage{listings}

\usepackage{lipsum} % Used for inserting dummy 'Lorem ipsum' text into the template

\usepackage{sectsty} % Allows customizing section commands
%\allsectionsfont{\normalfont\scshape} % Make all sections centered, the default font and small caps

\usepackage{fancyhdr} % Custom headers and footers
\pagestyle{fancyplain} % Makes all pages in the document conform to the custom headers and footers
\fancyhead{} % No page header - if you want one, create it in the same way as the footers below
\fancyfoot[L]{} % Empty left footer
\fancyfoot[C]{} % Empty center footer
\fancyfoot[R]{\thepage} % Page numbering for right footer
\renewcommand{\headrulewidth}{0pt} % Remove header underlines
\renewcommand{\footrulewidth}{0pt} % Remove footer underlines

\usepackage[compact]{titlesec}
\titlespacing{\section}{0pt}{*0}{*0}
\titlespacing{\subsection}{0pt}{*0}{*0}
\titlespacing{\subsubsection}{0pt}{*0}{*0}

\setlength{\parskip}{\baselineskip}%
\setlength{\parsep}{0pt}
\setlength{\headsep}{0pt}
\setlength{\topskip}{0pt}
\setlength{\topmargin}{0pt}
\setlength{\topsep}{0pt}
\setlength{\partopsep}{0pt}

%\numberwithin{equation}{section} % Number equations within sections (i.e. 1.1, 1.2, 2.1, 2.2 instead of 1, 2, 3, 4)
%\numberwithin{figure}{section} % Number figures within sections (i.e. 1.1, 1.2, 2.1, 2.2 instead of 1, 2, 3, 4)
%\numberwithin{table}{section} % Number tables within sections (i.e. 1.1, 1.2, 2.1, 2.2 instead of 1, 2, 3, 4)

\setlength\parindent{0pt} % Removes all indentation from paragraphs - comment this line for an assignment with lots of text

%----------------------------------------------------------------------------------------
%	TITLE SECTION
%----------------------------------------------------------------------------------------

\newcommand{\horrule}[1]{\rule{\linewidth}{#1}} % Create horizontal rule command with 1 argument of height

\title{	
\normalfont \normalsize 
\textsc{Parallel Computing I, Graduate Team Project} \\ % Your university, school and/or department name(s)
\horrule{2pt} \\[0.4cm] % Thin top horizontal rule
\vspace{-1.5em}
\huge Exhaustive Search Algorithms for $3$-$SAT$ \\ 
\Large Team Proposal \\
\vspace{-0.5em}
\horrule{2pt}
}
\author{Christopher Wood, Eitan Romanoff, Ankur Bajoria \\
Team Satisfaction\\
Website: {\tt http://ear7631.github.com/RIT-Parallel3Sat} } % Your name
\date{\large \today} % Today's date or a custom date

\begin{document}

\maketitle % Print the title

\vspace{-2em}

\begin{comment}
Describe how your parallel program is designed. As part of your description, address these points:
- Which parallel design patterns did you use?
- What data structures did you use?
- How did you partition the computation among the parallel threads?
- Did you need to synchronize the threads, and if so, how did you do it?
- Did you need to do load balancing, and if so, how did you do it?

Run your program with the first input data set posted below. Measure your program's running time on the parasite.cs.rit.edu SMP computer. Use the Parallel Java job queue running on the paragon.cs.rit.edu computer to do your timing measurements. To run in the job queue, your sequential program and your parallel program must include the statement Comm.init(args);. 
Measure your sequential program's running time by doing three program runs with the same input data. Measure your parallel program's running time for <NT> = 1, 2, 3, 4, and 8 parallel threads. For each value of <NT>, do three program runs with the same input data. Provide a table of your sequential and parallel program measurements with the following columns: number of threads; running time for first program run; running time for second program run; running time for third program run; smallest running time; measured speedup based on smallest running time; measured efficiency based on smallest running time; measured EDSF based on smallest running time.

Repeat Question 2 with the second input data set posted below.

Repeat Question 2 with the third input data set posted below.

Summarize your measurements from Questions 2-4. As part of your summary, address these points:
- How close to the ideal speedup and efficiency did your parallel program achieve as the number of parallel threads increased?
- What is causing the discrepancy if any between the ideal and the measured speedup and efficiency in your parallel program?
- Is this problem a good candidate for an SMP parallel program? Why or why not?

Write a paragraph telling me what you learned from this project.
\end{comment}

%%%%% REFERENCES %%%%%

\begin{thebibliography}{9}

\bibitem{algs} Gormen, Thomas H., Charles E. Leiserson, Ronald L. Rivest, and Clifford Stein. Introduction to Algorithms. MIT Press 44 (1990), 97-138.

\bibitem{dimacs} Satisfiability: Suggested Format, 1993. Available online at \\
{\tt http://people.sc.fsu.edu/\~jburkardt/pdf/dimacs\_cnf.pdf}. Accessed: 3/9/13.

\bibitem{paper1} Meyer, Quirin, Fabian Sch\"{o}nfeld, Marc Stamminger, and Rolf Wanka. 3-SAT on CUDA: Towards a Massively Parallel SAT Solver. \emph{2010 International Conference on High Performance Computing and Simulation (HPCS)}, IEEE (2010).

\bibitem{paper2} Hamadi, Youssef, Said Jabbour, and Lakhdar Sais. ManySAT: A Parallel SAT Solver. \emph{Journal on Satisfiability, Boolean Modeling and Computation}, 6.4 (2009), 245-262.

\bibitem{paper3} Zhang, Hantao, Maria Paola Bonacina, and Jieh Hsiang. PSATO: A Distributed Propositional Prover and its Application to Quasigroup Problems. \emph{Journal of Symbolic Computation} 21.4 (1996), 543-560.

\bibitem{satBook} Drechsler, Rolf and Stephan Eggersgl\"{u}b. High Quality Test Pattern Generation and Boolean Satisfiability. \emph{Springer}, 2012.

\bibitem{satCompetition} The International SAT Competitions Web Page. {\tt http://www.satcompetition.org/}. Accessed: 3/9/2013.

\bibitem{parallelComputing} Kaminsky, Alan. Building Parallel Programs: SMPs, Clusters, and Java. Cengage Course Technology, 2010. ISBN 1-4239-0198-3.

\end{thebibliography}

\end{document}