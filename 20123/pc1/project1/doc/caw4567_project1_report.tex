\documentclass[paper=a4, fontsize=11pt]{scrartcl} % A4 paper and 11pt font size
\usepackage[left=2cm,top=1cm,right=2cm,nohead,nofoot]{geometry}
\usepackage[T1]{fontenc} % Use 8-bit encoding that has 256 glyphs
\usepackage{fourier} % Use the Adobe Utopia font for the document - comment this line to return to the LaTeX default
\usepackage[english]{babel} % English language/hyphenation
\usepackage{amsmath,amsfonts,amsthm} % Math packages
\usepackage{listings}

\usepackage{lipsum} % Used for inserting dummy 'Lorem ipsum' text into the template

\usepackage{sectsty} % Allows customizing section commands
%\allsectionsfont{\normalfont\scshape} % Make all sections centered, the default font and small caps

\usepackage{fancyhdr} % Custom headers and footers
\pagestyle{fancyplain} % Makes all pages in the document conform to the custom headers and footers
\fancyhead{} % No page header - if you want one, create it in the same way as the footers below
\fancyfoot[L]{} % Empty left footer
\fancyfoot[C]{} % Empty center footer
\fancyfoot[R]{\thepage} % Page numbering for right footer
\renewcommand{\headrulewidth}{0pt} % Remove header underlines
\renewcommand{\footrulewidth}{0pt} % Remove footer underlines

\usepackage[compact]{titlesec}
\titlespacing{\section}{0pt}{*0}{*0}
\titlespacing{\subsection}{0pt}{*0}{*0}
\titlespacing{\subsubsection}{0pt}{*0}{*0}

\setlength{\parskip}{\baselineskip}%
\setlength{\parsep}{0pt}
\setlength{\headsep}{0pt}
\setlength{\topskip}{0pt}
\setlength{\topmargin}{0pt}
\setlength{\topsep}{0pt}
\setlength{\partopsep}{0pt}

%\numberwithin{equation}{section} % Number equations within sections (i.e. 1.1, 1.2, 2.1, 2.2 instead of 1, 2, 3, 4)
%\numberwithin{figure}{section} % Number figures within sections (i.e. 1.1, 1.2, 2.1, 2.2 instead of 1, 2, 3, 4)
%\numberwithin{table}{section} % Number tables within sections (i.e. 1.1, 1.2, 2.1, 2.2 instead of 1, 2, 3, 4)

\setlength\parindent{0pt} % Removes all indentation from paragraphs - comment this line for an assignment with lots of text

%----------------------------------------------------------------------------------------
%	TITLE SECTION
%----------------------------------------------------------------------------------------

\newcommand{\horrule}[1]{\rule{\linewidth}{#1}} % Create horizontal rule command with 1 argument of height

\title{	
\normalfont \normalsize 
\textsc{Parallel Computing I, Graduate Team Project} \\ % Your university, school and/or department name(s)
\horrule{2pt} \\[0.4cm] % Thin top horizontal rule
\vspace{-1.5em}
\huge Exhaustive Search Algorithms for $3$-$SAT$ \\ 
\Large Team Proposal \\
\vspace{-0.5em}
\horrule{2pt}
}
\author{Christopher Wood, Eitan Romanoff, Ankur Bajoria \\
Team Satisfaction\\
Website: {\tt http://ear7631.github.com/RIT-Parallel3Sat} } % Your name
\date{\large \today} % Today's date or a custom date

\begin{document}

\maketitle % Print the title

\vspace{-2em}
\section{Problem Description}
For our project we choose $3$-$SAT$ (or, more formally, $3$-$CNF$-$SAT$), which 
is an NP-complete decision problem \cite{algs}. $3$-$SAT$ takes as input 
a $3$-$CNF$ Boolean formula and returns YES if the formula is 
satisfiable, and NO otherwise. A 3-CNF formula $\phi_n$ on $n$ variables
is expressed as the conjunction (Boolean AND) of arbitrarily many clauses, 
where each clause is the disjunction (Boolean OR) of exactly three literals (a 
literal is a Boolean variable or its negation). A formula $\phi_n$ 
is said to be satisfiable if and only if there exists an 
assignment of truth values to the $n$ variables such that substituting them 
into the literals of $\phi$ will cause it to evaluate to true. 
Expressed as a formal language, we have 
$3$-$SAT = \{\langle \phi \rangle : \phi \text{ is a satisfiable }3\text{-}CNF\text{ formula} \}$
(i.e. the set of all $3$-$CNF$ formula strings that are accepted by the 
$3$-$SAT$ language if they are satisfiable).

\section{Exhaustive Search Algorithm for $3$-$SAT$}
An exhaustive search algorithm for solving the $3$-$SAT$ 
problem with input $\phi_n$ iterates over 
all $2^n$ configurations of variable truth values, and for each 
configuration, assigns the truth value of each variable 
to the appropriate literal in $\phi_n$, and then evaluates $\phi_n$ to determine if it is true. If for 
every possible variable configuration $\phi_n$ does not evaluate to 
true, then the exhaustive $3$-$SAT$ algorithm returns NO. Otherwise, 
some satisfiable truth value configuration must exist, and so the exhaustive $3$-$SAT$ algorithm 
returns YES. 

\section{Programs and Performance Metrics}
The sequential and parallel programs we will deliver will take as input 
a $3$-$CNF$ formula $\phi_n$, encoded using the DIMACS $CNF$ format \cite{dimacs}, 
and output a Boolean truth value indicating whether or not the 
formula is satisfiable. The $3$-$CNF$ formula will be entered at the command 
line or it will be read from a file to facilitate our experiments. Based on 
the $3$-$SAT$ problem, the only restriction is that each clause must have 
exactly three literals. Therefore, our programs will enable the 
number of clauses and the number of variables 
to be parameters defined in the DIMACS $CNF$ format. An example of the 
$3$-$CNF$ formula $(x_1 \lor x_2 \lor \lnot x_3)$, which has one clause and three 
variables, encoded using DIMACS is shown below.
\begin{center}
\begin{lstlisting}[frame=single]
p cnf 3 1
1 2 -3 0
\end{lstlisting}
\end{center}
\vspace{-1em}
There are two main parts of the programs that implement the exhaustive search algorithm outlined
in Section 2. First, the program must read in the $3$-$CNF$ formula $\phi_n$ and set up 
the appropriate data structures. Second, the program must traverse 
and assign all possible configurations of variable truth 
values into $\phi_n$ to then check for satisfiability. Both the 
sequential and parallel programs will share the first part, which
is an inherently sequential task, so as to set up 
the shared data structures containing the $3$-$CNF$ formula.

The second part of the program can be done sequentially or in parallel. 
In a parallel program, this part is an instance of an agenda parallel problem, where each 
task evaluates a single truth value assignment for a variable configuration to $\phi_n$ and returns the Boolean result.
At the end of the program we are only concerned with the summary of the task results (i.e. whether or not
one task returned true). 
Furthermore, since there are no sequential dependencies that exist between each agenda task, we can divvy 
up the execution of these $2^n$ tasks among $2^n$ virtual processors. When 
implemented, we will clump the execution of many tasks together on a single 
processor because we will not have $2^n$ processors available for computation for large $n$. 

Finally, since $3$-$SAT$ is both an interesting problem in academia and often arises in 
the industry, we will be measuring the metrics of speedup and sizeup as functions
of the program running time $T$, problem size $N$, and number of processors $K$. In addition, we will also measure 
the efficiency and sizeup efficiency metrics. To compute the speedup and efficiency metrics, we will
measure the execution time of the sequential ($T_{seq}(N,K), K = 1$) and parallel ($T_{par}(N,K)$) programs. We will also experimentally determine the sequential fraction of 
the program using these execution time values. Similarly, to compute the sizeup and sizeup efficiency metrics,
we will measure the problem size for the sequential ($N_{seq}(T,K)$) and parallel ($N_{seq}(T,K)$)
programs. However, as recognized by Gustafson, the First Problem Size Law for $N(T,K)$ is
merely an approximation made under the assumption that the sequential fraction 
of the program execution time is constant for all problem sizes $N$, which is not necessarily true. Therefore, we 
will compute the sizeup and sizeup efficiency by measuring $N(T,K)$ under the Second Problem Size Law \cite{parallelComputing}.

\section{Literature Survey and Additional Resources}
As part of the graduate requirement, we will analyze \cite{paper1}, \cite{paper2},
and \cite{paper3} in our literature survey. In addition, we will use the information and resources made available
on the International $SAT$ Competition web site \cite{satCompetition}, as well
as the book by Drechsler et al. on advanced $SAT$ solving techniques \cite{satBook}. 

%%%%% REFERENCES %%%%%

\begin{thebibliography}{9}

\bibitem{algs} Gormen, Thomas H., Charles E. Leiserson, Ronald L. Rivest, and Clifford Stein. Introduction to Algorithms. MIT Press 44 (1990), 97-138.

\bibitem{dimacs} Satisfiability: Suggested Format, 1993. Available online at \\
{\tt http://people.sc.fsu.edu/\~jburkardt/pdf/dimacs\_cnf.pdf}. Accessed: 3/9/13.

\bibitem{paper1} Meyer, Quirin, Fabian Sch\"{o}nfeld, Marc Stamminger, and Rolf Wanka. 3-SAT on CUDA: Towards a Massively Parallel SAT Solver. \emph{2010 International Conference on High Performance Computing and Simulation (HPCS)}, IEEE (2010).

\bibitem{paper2} Hamadi, Youssef, Said Jabbour, and Lakhdar Sais. ManySAT: A Parallel SAT Solver. \emph{Journal on Satisfiability, Boolean Modeling and Computation}, 6.4 (2009), 245-262.

\bibitem{paper3} Zhang, Hantao, Maria Paola Bonacina, and Jieh Hsiang. PSATO: A Distributed Propositional Prover and its Application to Quasigroup Problems. \emph{Journal of Symbolic Computation} 21.4 (1996), 543-560.

\bibitem{satBook} Drechsler, Rolf and Stephan Eggersgl\"{u}b. High Quality Test Pattern Generation and Boolean Satisfiability. \emph{Springer}, 2012.

\bibitem{satCompetition} The International SAT Competitions Web Page. {\tt http://www.satcompetition.org/}. Accessed: 3/9/2013.

\bibitem{parallelComputing} Kaminsky, Alan. Building Parallel Programs: SMPs, Clusters, and Java. Cengage Course Technology, 2010. ISBN 1-4239-0198-3.

\end{thebibliography}

\end{document}