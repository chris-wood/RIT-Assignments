\documentclass[11pt]{article}

\usepackage{thumbpdf, amssymb, amsmath, amsthm, microtype,
	    graphicx, verbatim, listings, color, fancybox}
\usepackage[pdftex]{hyperref}
%\usepackage[margin=1in]{geometry}
\usepackage{cawsty}
\usepackage{fullpage}
\usepackage{pseudocode}

\newcommand{\basiseq}{
\begin{bmatrixr}{2}
    I & \vec{0}\\
    A & -B
  \end{bmatrixr}
  \begin{bmatrixr}{1}
    U\\ 1
  \end{bmatrixr}
  =
  \begin{bmatrixr}{1}
    U\\ \vec{0}
  \end{bmatrixr}
}

\newcommand{\sbasiseq}{
\big[\begin{smallmatrix}
    I & \vec{0}\\
    A & -B
  \end{smallmatrix}\big]
  \big[\begin{smallmatrix}
    U \\ 1
  \end{smallmatrix}\big]
  =\big[
  \begin{smallmatrix}
    U \\ \vec{0}
  \end{smallmatrix}\big]
}

%\setlength{\parindent}{0pt}

\linespread{1.2}

\begin{document}
\arltitle{4005-800 Algorithms}{Homework 1}
\begin{prob}{8.1}Give a complete proof of Lemma 8.1: $U$ solves $AU=B$ if and only if $U$ solves
\begin{equation*}\basiseq\end{equation*}
\end{prob}
\begin{sol} Let $A$ be an $m$ by $n$ matrix and $B=[b_1\;b_2\ldots b_m]^T$
  \begin{enumalph}
    \item {\it Claim:} If $U=[u_1\; u_2 \ldots u_n]^T$ such that $AU=B$, then $\sbasiseq$\\
      {\it Proof:} Since $AU=B$, we know that
      \begin{align*}
        a_{11}u_1+a_{12}u_2&+\dots+a_{1n}u_n = b_1\\
        a_{21}u_1+a_{22}u_2&+\dots+a_{2n}u_n = b_2\\
        &\quad\vdots\\
        a_{m1}u_1+a_{m2}u_2&+\dots+a_{mn}u_n = b_m
      \end{align*}
      The first $n$ rows of $\big[\begin{smallmatrix}I & \vec{0}\\A & -B\end{smallmatrix}\big]$ make up $[\;I\;\;\vec{0}\;]$ and clearly,
      \begin{equation*}
        \begin{bmatrixr}{2}
          I & \vec{0}
        \end{bmatrixr}
        \begin{bmatrixr}{1}
          U\\ 1
        \end{bmatrixr}
        = U.
      \end{equation*}
      The last $m$ rows make up $[\;A\;\;-B\;]$, so
      \begin{align*}
        \begin{bmatrixr}{2}
          A & -B
        \end{bmatrixr}
        \begin{bmatrixr}{1}
          U\\ 1
        \end{bmatrixr}
        &=
        \begin{bmatrix}
          (a_{11}u_1+a_{12}u_2+\dots+a_{1n}u_n) - b_1\\
          (a_{21}u_1+a_{22}u_2+\dots+a_{2n}u_n) - b_2\\
          \vdots\\
          (a_{m1}u_1+a_{m2}u_2+\dots+a_{mn}u_n) - b_m
        \end{bmatrix}\\
        &=\begin{bmatrix}b_1 - b_1\\b_2 - b_2\\\vdots\\b_m - b_m\end{bmatrix}\\
        &= \vec{0}
      \end{align*}
      Therefore, $\sbasiseq$.
    \item {\it Claim:} If $U=[u_1\; u_2 \ldots u_n]^T$ such that $\sbasiseq$, then $AU=B$. \\
      {\it Proof:} Very similar to part (a), but in reverse.
  \end{enumalph}
\end{sol}

\begin{prob}{8.2}
  Which of the following are in the lattice $\mathcal{L}=\laspan_\bbZ(B)$ where
  \begin{equation*}
    B=\begin{bmatrixr}{2}1 &-2\\3 &1\end{bmatrixr}
  \end{equation*}
\end{prob}
\begin{sol} Let $\vec{b_1}=[1,3]$ and $\vec{b_2}=[-2,1]$ (the columns of $B$). The following vectors are in $\mathcal{L}$
  \begin{itemize}
    \item $5\vec{b_1}+3\vec{b_2}=[-1,18] \implies [-1,18]\in\mathcal{L}$
    \item $4\vec{b_1}=[4,12] \implies [4,12]\in\mathcal{L}$
    \item $3\vec{b_1}+\vec{b_2}=[1,10] \implies [1,10]\in\mathcal{L}$
    \item $-3\vec{b_1}+2\vec{b_2}=[1,-11] \implies [1,-11]\in\mathcal{L}$
  \end{itemize}
\end{sol}

\begin{prob}{8.3} Compute $\mathsf{wt}(M)$ and $\mathsf{vol}({\mathcal{L}})$ with $M=\big[\begin{smallmatrix}1&-2\\3&1\end{smallmatrix}\big]$
and verify Hadamard's inequality for this lattice.\end{prob}

\begin{sol} We first apply {\sc Gram-Schmidt} to $M$. Since there's only two vectors ($\vec{b_1}$ and $\vec{b_2}$), we know $\vec{b_1^*}=\vec{b_1}$ and we only need to compute $\vec{b_2^*}$
  \begin{equation*}
    \vec{b_2^*}=\vec{b_2}-\frac{\vec{b_1}\cdot\vec{b_2}}{\vec{b_1}\cdot\vec{b_1}}\vec{b_1}=
    \begin{bmatrixr}{1}-2\\1\end{bmatrixr}-\frac{1}{10}\begin{bmatrixr}{1}1\\3\end{bmatrixr}=\begin{bmatrixr}{1}-\frac{21}{10}\\\tfrac{7}{10}\end{bmatrixr}
  \end{equation*}
  We can then compute $\mathsf{wt}(M)$ and $\mathsf{vol}({\mathcal{L}})$
  \begin{align*}
    \mathsf{wt}(M) &= \norm{\vec{b_1}} \cdot \norm{\vec{b_2}} \\
      &= \sqrt{10}\cdot\sqrt{5}\\
      &= 5\sqrt{2}\\&\\
    \mathsf{vol}({\mathcal{L}}) &= \norm{\vec{b_1^*}} \cdot \norm{\vec{b_2^*}} \\
      &= \sqrt{10}\cdot \sqrt{\frac{21^2+7^2}{10^2}}\\
      &= 7
  \end{align*}
  Since $\sqrt{2}\approx 1.4142 > 7/5=1.4$, we know that $\mathsf{wt}(M)>\mathsf{vol}(\mathcal{L})$, which agrees with {\it Hadamard's inequality}.
\end{sol}

\begin{prob}{8.5}Using the Gram-Schmidt algorithm, work out the orthogonal basis for the lattice spanned by
\begin{equation*}
  M=\begin{bmatrixr}{3}1&1&2\\1&0&1\\1&1&0\end{bmatrixr}
\end{equation*}
\end{prob}
\begin{sol} Let $\vec{b_1}=[1,1,1]$, $\vec{b_2}=[1,0,1]$ and $\vec{b_3}=[2,1,0]$. Then $\vec{b_1^*}=\vec{b_1}$,
  \begin{align*}
    \vec{b_2^*} &= \vec{b_2}-\frac{\vec{b_1^*}\cdot \vec{b_2}}{\vec{b_1^*} \cdot \vec{b_1^*}}\vec{b_1^*}\\
      &= [1,0,1] - \tfrac{2}{3}[1,1,1]\\
      &= [\tfrac{1}{3}, -\tfrac{2}{3}, \tfrac{1}{3}]\\&\\
    \vec{b_3^*} &= \vec{b_3}-\frac{\vec{b_1^*}\cdot \vec{b_3}}{\vec{b_1^*} \cdot \vec{b_1^*}}\vec{b_1^*} - \frac{\vec{b_2^*}\cdot \vec{b_3}}{\vec{b_2^*} \cdot \vec{b_2^*}}\vec{b_2^*}\\
      &= [2,1,0] - \tfrac{3}{3}[1,1,1] - 0[\tfrac{1}{3}, -\tfrac{2}{3}, \tfrac{1}{3}]\\
      &= [1,0,-1]
  \end{align*}
  We can now compute $\mathsf{wt}(M)=\norm{\vec{b_1}}\cdot\norm{\vec{b_2}}\cdot\norm{\vec{b_3}}=\sqrt{30}$ and $\mathsf{vol}({\mathcal{L}})=\norm{\vec{b_1^*}}\cdot\norm{\vec{b_2^*}}\cdot\norm{\vec{b_3^*}}=2$. Clearly, $\mathsf{wt}(M)>\mathsf{vol}(\mathcal{L})$.
\end{sol}

\begin{prob}{8.6}
  Consider the matrix
  \begin{equation*}
    M=\begin{bmatrixr}{4}
    0 & 2 & 3 & 1 \\
    1 & 0 & -1 & 5\\
    2 & -2 & 2 & 0\\
    2 & 2 & 0 & -2
    \end{bmatrixr}
  \end{equation*}
  (a) Show that $M$ is a reduced matrix (b) Verify the inequalities of a reduced matrix hold for $M$
\end{prob}
\begin{sol} We let $\vec{b_1}=[0,1,2,2]$, $\vec{b_2}=[2,0,-2,2]$, $\vec{b_3}=[3,-1,2,0]$ and $\vec{b_4}=[1,5,0,-2]$. We then run {\sc GramSchmidt}$(\vec{b_1},\vec{b_2},\vec{b_3},\vec{b_4})$ and obtain $\vec{b_1^*}=[0,1,2,2]$, $\vec{b_2^*}=[2,0,-2,2]$, $\vec{b_3^*}=[2.6667,-1.3333,1.6667,-1]$ and $\vec{b_4^*}=[1.7544,4.6784,-0.2924,-2.0468]$.
  \begin{enumalph}
    \item In order to show that $M$ is a reduced basis, we first need all $|\alpha_{i,j}|<\tfrac{1}{2}$ for all $i<j$. These values can be represented as an upper triangular matrix with all zeros in the diagonal:
      \begin{equation*}
        \alpha = \begin{bmatrixr}{4}
          & 0 & 0.3333 & 0.1111\\
          & & 0.1667 & -0.1667\\
          & & & -0.1579 \\
          0 & & &
        \end{bmatrixr}
      \end{equation*}
      Clearly, the absolute value of the six appropriate entries of $\alpha$ are all less than $\tfrac{1}{2}$.
    
      Next, we need to show that for all $j=1,2,\dots,n-1$,
      \begin{equation*}
        \norm{\vec{b_{j+1}^*} + \alpha_{j,j+1}\vec{b_j^*}}^2 \geq \tfrac{3}{4}\norm{\vec{b_j^*}}^2
      \end{equation*}
      \begin{align*}
        j=1,&  &\norm{\vec{b_{2}^*} + \alpha_{1,2}\vec{b_1^*}}^2=12,\;\;\tfrac{3}{4}\norm{\vec{b_1^*}}^2&=\tfrac{3}{4}\cdot9\\
        j=2,&  &\norm{\vec{b_{3}^*} + \alpha_{2,3}\vec{b_2^*}}^2=13,\;\;\tfrac{3}{4}\norm{\vec{b_2^*}}^2&=\tfrac{3}{4}\cdot12\\
        j=3,&  &\norm{\vec{b_{4}^*} + \alpha_{3,4}\vec{b_3^*}}^2=29.556,\;\;\tfrac{3}{4}\norm{\vec{b_3^*}}^2&=\tfrac{3}{4}\cdot12.6679
      \end{align*}
    \item Next we need to show the following two results:
      \begin{enumerate}
        \item $\norm{\vec{b_1}}\leq 2^{(n-1)/4}\mathsf{vol}({\mathcal{L}})^{1/n}$
        \item $\mathsf{wt}(M) \leq 2^{n(n-1)/4}\mathsf{vol}({\mathcal{L}})$
      \end{enumerate} We compute $\mathsf{wt}(M)=\sqrt{9\cdot12\cdot14\cdot30}\approx212.9789$ and $\mathsf{vol}({\mathcal{L}})\approx199.9999$. Then $\norm{\vec{b_1}}=3$ and $2^{3/4}\mathsf{vol}({\mathcal{L}})^{1/4}=6.3246$ which implies (1) is true. We compute $2^{3}\mathsf{vol}({\mathcal{L}})=1599.9996$ which shows (2) to be true.
  \end{enumalph}
\end{sol}

\end{document}
